%        File: arfc-beamer.tex
%     Created: Sun May 5 10:00 PM 2013 C
%


%\documentclass[11pt,handout]{beamer}
\documentclass[9pt]{beamer}
\usetheme[white]{Illinois}
%\title[short title]{long title}
\title[Modeling Without Equilibrium]{LEU+ to HALEU transitions in advanced reactor fuel cycles}
%\subtitle[short subtitle]{long subtitle}
\subtitle[Short SubTitle]{ORNL Symposium}
%\author[short name]{long name}
\author[Nathan Ryan]{Nathan Ryan\\Advanced Reactors and Fuel Cycles}
%\date[short date]{long date}
\date[03.08.2025]{March 8, 2025}
%\institution[short name]{long name}
\institute[UIUC]{University of Illinois Urbana-Champaign}

%\usepackage{bbding}
\usepackage{amsfonts}
% \usepackage{algorithm}
% \usepackage[ruled]{algorithm2e}
% \usepackage{algorithmic}
\usepackage{algpseudocode}
% \usepackage{algorithmic}
% \usepackage{array}
\usepackage{amsmath}
\usepackage{xspace}
\usepackage{graphicx}
\usepackage{subfigure}
\usepackage{booktabs} % nice rules for tables
\usepackage{microtype} % if using PDF
\usepackage{bigints}
\usepackage{caption}

\newcommand{\units}[1] {\:\text{#1}}%
\newcommand{\SN}{S$_N$}%{S$_\text{N}$}%{$S_N$}%
\DeclareMathOperator{\erf}{erf}
%I need some complimentary error funcitons...
\DeclareMathOperator{\erfc}{erfc}
%Those icons in the references are terrible looking
\setbeamertemplate{bibliography item}[text]

%%%% Acronym support

\usepackage[acronym,toc]{glossaries}
\include{acros}

\makeglossaries

%try to get rid of header on title page\dots
\makeatletter
    \newenvironment{withoutheadline}{
        \setbeamertemplate{headline}[default]
        \def\beamer@entrycode{\vspace*{-\headheight}}
    }{}
\makeatother

% \makeatother
% \setbeamertemplate{footline}
% {
%   \leavevmode%
%   \hbox{%
%     \rightline{\insertframenumber{} / \inserttotalframenumber\hspace*{1ex}}
%   }%
%   \vskip0pt%
% }
% \makeatletter
\setbeamertemplate{caption}{\raggedright\insertcaption\par}
\setbeamertemplate{page number in head/foot}[appendixframenumber]

\begin{document}
%%%%%%%%%%%%%%%%%%%%%%%%%%%%%%%%%%%%%%%%%%%%%%%%%%%%%%%%%%%%%
%% From uw-beamer Here's a handy bit of code to place at
%% the beginning of your presentation (after \begin{document}):
\newcommand*{\alphabet}{ABCDEFGHIJKLMNOPQRSTUVWXYZabcdefghijklmnopqrstuvwxyz}
\newlength{\highlightheight}
\newlength{\highlightdepth}
\newlength{\highlightmargin}
\setlength{\highlightmargin}{2pt}
\settoheight{\highlightheight}{\alphabet}
\settodepth{\highlightdepth}{\alphabet}
\addtolength{\highlightheight}{\highlightmargin}
\addtolength{\highlightdepth}{\highlightmargin}
\addtolength{\highlightheight}{\highlightdepth}
\newcommand*{\Highlight}{\rlap{\textcolor{HighlightBackground}{\rule[-\highlightdepth]{\linewidth}{\highlightheight}}}}
%%%%%%%%%%%%%%%%%%%%%%%%%%%%%%%%%%%%%%%%%%%%%%%%%%%%%%%%%%%%%
%%--------------------------------%%
\begin{withoutheadline}
\frame{
  \titlepage
}
\end{withoutheadline}

%%--------------------------------%%
\AtBeginSection[]{
\begin{frame}
  \frametitle{Outline}
  \tableofcontents[currentsection]
\end{frame}
}

\section{My Background}
  \begin{frame}
    \begin{columns}
      \column[t]{5cm}
      Sometimes things need to be put side by side, in two nice
      looking columns.

      Maybe one column involves a quotation.

      \begin{quote}
              Explicit is better than implicit. -- The Zen of Python
      \end{quote}


      And, also, perhaps, a logo.
      \begin{center}
              \includegraphics[height=0.2\textheight]{./images/arfc-logo}
      \end{center}
      \column[t]{5cm}
        \begin{figure}[htbp!]
          \begin{center}
            \includegraphics[height=4cm]{./images/kitten}
          \end{center}
          \caption{A caption describing the image. \cite{lastname_firstword_1900}.}
          \label{fig:kittenfigure}
        \end{figure}
    \end{columns}
  \end{frame}


\section{Nuclear Fuel Cycle}
  \begin{frame}
      \frametitle{Generally, fuel cycles have these steps}
      \begin{figure}[ht!]
      \centering
      \includegraphics[width=0.75\textwidth]{images/nuclear_fuel_cycle.png}
      \caption{Source: Penn State Univ. Radiation Science and Engineering Center (public domain)$^{*}$}
      \end{figure}
  \end{frame}

  \begin{frame}
      \frametitle{Not all fuel cycles are made equal, and we want options}
      Concerns about economics, waste generation, proliferation risk, and sustainability motivate the need for fuel cycle options. With metrics like:
        \begin{itemize}%[<+->]
            \item natural resource utilization,
            \item waste mass/volume,
            \item special material quantities,
            \item separative work units,
            \item and energy production,
        \end{itemize}
        we can begin to evaluate the tradeoffs between fuel cycle options.
  \end{frame}

\subsection{Fuel Cycle Modeling}
%   \begin{frame}
%     \frametitle{Big questions in fuel cycle modeling}
%     Increased computational power and advanced reactors mean more detailed fuel cycle modeling.
%     \begin{itemize}
%         \item How can we make facility models more accurate?
%         \item How can we make transaction models more detailed?
%         \item Can we implement nuclear fuel cycle codes to identify realtime diversion or diversion paths?
%         \item When do advanced reactor technologies change key metrics we use to evaluate fuel cycles?
%     \end{itemize}
%   \end{frame}

  \begin{frame}
    \frametitle{We use Cyclus to model fuel cycles}
    \vspace{20pt}
    Cyclus is an open-source agent-based fuel cycle code allowing for detailed facility and transaction modeling \cite{huff_fundamental_2016}.
    \vspace{20pt}
    \begin{figure}
        \centering
        \includegraphics[width=0.45\textwidth]{images/cyclus_logo.png}
    \end{figure}

    \vspace{37pt}
    Source: \url{https://github.com/cyclus/cyclus.github.com/blob/source/source/logos/logo2_transp.png}
  \end{frame}

  \begin{frame}
    \frametitle{Cyclus is being used to tackle big questions in fuel cycle modeling}
    \begin{block}{Making facility models more accurate}
        OpenMCyclus \cite{openmcyclus_paper} couples Cyclus with OpenMC to model realtime depletion.
    \end{block}
    \begin{block}{Making transaction models more detailed}
        There is active work to incorporate realistic purchasing agreements and market models into Cyclus.
    \end{block}
    \begin{block}{Identifying realtime diversion or diversion paths}
        CNTAUR \cite{mummah_advanced_2024} and Pyre \cite{westphal_modeling_2019} format outputs in IAEA code 10 format and model real time diversion, respectively.
    \end{block}
    \begin{block}{Finding advanced reactor impacts on the fuel cycle}
        We will talk about that today!
    \end{block}
  \end{frame}

  \section{Deployment Schemes}
  \begin{frame}
    \frametitle{Greedy reactor deployment algorithm}
    % \begin{algorithm}[H]
      \begin{algorithmic}[1]
        % \caption{Greedy Reactor Deployment Algorithm}
          \State Initialize demand
          \While{demand exists}
              \State Select the largest reactor that does not exceed demand
              \State Deploy reactors until the next reactor exceeds demand
              \State Update demand
          \EndWhile
      \end{algorithmic}
    % \end{algorithm}
  \end{frame}

  \begin{frame}
    \frametitle{Random reactor deployment algorithm}
  %   \begin{algorithm}[H]
  %     \caption{Random Reactor Deployment Algorithm}
      \begin{algorithmic}[1]
          \State Initialize demand
          \While{demand exists}
              \State Randomly deploy a reactor that does not exceed demand
              \State Update demand
          \EndWhile
      \end{algorithmic}
  %     \end{algorithm}
  \end{frame}

  \begin{frame}
    \frametitle{Random + greedy reactor deployment algorithm}
  %     \begin{algorithm}[H]
  %       \caption{Random + Greedy Reactor Deployment Algorithm}
        \begin{algorithmic}[1]
            \State Initialize demand
            \While{demand exists}
                \State Randomly deploy a reactor
                \If{demand is exceeded}
                    \State Remove last reactor
                    \If{demand still exists}
                        \State Select the largest reactor that does not exceed demand
                        \State Deploy until the next reactor exceeds demand
                        \State Update demand
                    \EndIf
                \EndIf
            \EndWhile
        \end{algorithmic}
  %       \end{algorithm}
  \end{frame}

  \section{LEU+ to HALEU}
  \begin{frame}
    \frametitle{What if we can't get HALEU to fuel these advanced reactors?}
    \vspace{-25pt}
    \begin{figure}
        \centering
        \includegraphics[width=0.98\textwidth]{images/reactor_timeline.png}
        \caption{Source: \url{inl.gov/nuclear-reactor-sustainment-and-expanded-deployment/}}
    \end{figure}
    \vspace{-8pt}
    Could we use LEU+ in the meantime?
  \end{frame}

  \begin{frame}
    \frametitle{We define the enrichment levels as...}
    These are a mash-up of economic and regulatory definitions.
    \begin{table}[H]
        \centering
        \caption{Enrichment levels and their ranges.}
        \label{tab:enrichment_levels}
        \begin{tabular}{c c}
           \hline
           \textbf{Enrichment Level} & \textbf{Range [\%  $^{235}$U]} \\
           \hline
           Natural & $<$ 0.711 \\
           LEU & 0.711-5 \\
           LEU+ & 5-10 \\
           HALEU & 10-20 \\
           HEU & $\geq$ 20  \\
           \hline
        \end{tabular}
     \end{table}
  \end{frame}

  \begin{frame}
    \frametitle{Our demand for energy is going up}
    \begin{figure}
        \centering
        \includegraphics[width=0.85\textwidth]{images/new_capacity_ng_d2.pdf}
    \end{figure}
  \end{frame}

  \begin{frame}
    \frametitle{Staggering enrichment could give the supply chain time to develop}
    \begin{figure}
        \centering
        \includegraphics[width=0.75\textwidth]{images/fresh_fuel.pdf}
    \end{figure}
  \end{frame}

  \begin{frame}
    \frametitle{The difference is on the order of hundreds of tons}
    \begin{figure}
        \centering
        \includegraphics[width=0.75\textwidth]{images/fresh_fuel_difference.pdf}
    \end{figure}
  \end{frame}


  \section{Conclusion}
  \begin{frame}
      \frametitle{Fuel cycles modeling is useful for enegy planning and safeguards}
      We have covered a tiny fraction of what fuel cycle modeling can do, but there is so much more to do. In our simple case, we transition from LEU+ to HALEU after 10 years of operation.
      \begin{itemize}
          \item For the Xe100 reactors, we need almost 315 less tons of HALEU.
          \item For the MMR reactors, we need almost 97 less tons of HALEU.
      \end{itemize}
      Next we need to characterize what the cost of this transition would be.
  \end{frame}


\begin{frame}
  \frametitle{Acknowledgements}
  This research was performed, in part, using funding received from the DOE
  Office of Nuclear Energy's Nuclear Energy University Program (Project 23-29656
  DE-NE0009390) 'Illuminating Emerging Supply Chain and Waste Management
  Challenges'.
  \vspace{0.5cm}
  This research was supported in part by an appointment to the Oak Ridge
  National Laboratory Research Student Internships Program, sponsored by the U.
  S. Department of Energy and administered by the Oak Ridge Institute for
  Science and Education.
\end{frame}




%%--------------------------------%%
%%--------------------------------%%
\begin{frame}[allowframebreaks]
  \frametitle{References}
  \bibliographystyle{plain}
  {\footnotesize \bibliography{bibliography.bib} }

\end{frame}

\appendix

\begin{frame}
    \frametitle{Know how to code?}
    Consider volunteering as a TA or mentor in the Computational Resource Access NEtwork (CRANE) so we can support more students!
    \begin{figure}
        \centering
        \includegraphics[width=0.7\textwidth]{images/CRANE_logo_inverted.png}
    \end{figure}
    Go to our website: \url{https://www.cranephysics.org}
\end{frame}

%%--------------------------------%%


\end{document}
