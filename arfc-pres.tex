%        File: arfc-beamer.tex
%     Created: Sun May 5 10:00 PM 2013 C
%


%\documentclass[11pt,handout]{beamer}
\documentclass[9pt]{beamer}
\usetheme[white]{Illinois}
%\title[short title]{long title}
\title[Updating Fuel Cycle Assumptions]{Updates to time, power, and deployment in advanced reactor fuel cycle modeling}
%\subtitle[short subtitle]{long subtitle}
\subtitle[Short SubTitle]{ORNL Symposium}
%\author[short name]{long name}
\author[Nathan Ryan]{Nathan Ryan\\Advanced Reactors and Fuel Cycles}
%\date[short date]{long date}
\date[03.27.2025]{March 27, 2025}
%\institution[short name]{long name}
\institute[UIUC]{University of Illinois Urbana-Champaign}

%\usepackage{bbding}
\usepackage{amsfonts}
% \usepackage{algorithm}
% \usepackage[ruled]{algorithm2e}
% \usepackage{algorithmic}
\usepackage{algpseudocode}
% \usepackage{algorithmic}
% \usepackage{array}
\usepackage{amsmath}
\usepackage{xspace}
\usepackage{graphicx}
\usepackage{subfigure}
\usepackage{booktabs} % nice rules for tables
\usepackage{microtype} % if using PDF
\usepackage{bigints}
\usepackage{caption}

\newcommand{\cycamore}{\textsc{Cycamore}\xspace}
\newcommand{\cyclus}{\textsc{Cyclus}\xspace}

\newcommand{\units}[1] {\:\text{#1}}%
\newcommand{\SN}{S$_N$}%{S$_\text{N}$}%{$S_N$}%
\DeclareMathOperator{\erf}{erf}
%I need some complimentary error funcitons...
\DeclareMathOperator{\erfc}{erfc}
%Those icons in the references are terrible looking
\setbeamertemplate{bibliography item}[text]

%%%% Acronym support

\usepackage[acronym,toc]{glossaries}
\include{acros}

\makeglossaries

%try to get rid of header on title page\dots
\makeatletter
    \newenvironment{withoutheadline}{
        \setbeamertemplate{headline}[default]
        \def\beamer@entrycode{\vspace*{-\headheight}}
    }{}
\makeatother

% \makeatother
% \setbeamertemplate{footline}
% {
%   \leavevmode%
%   \hbox{%
%     \rightline{\insertframenumber{} / \inserttotalframenumber\hspace*{1ex}}
%   }%
%   \vskip0pt%
% }
% \makeatletter
\setbeamertemplate{caption}{\raggedright\insertcaption\par}
\setbeamertemplate{page number in head/foot}[appendixframenumber]
\setbeamertemplate{caption}[numbered]

\begin{document}
%%%%%%%%%%%%%%%%%%%%%%%%%%%%%%%%%%%%%%%%%%%%%%%%%%%%%%%%%%%%%
%% From uw-beamer Here's a handy bit of code to place at
%% the beginning of your presentation (after \begin{document}):
\newcommand*{\alphabet}{ABCDEFGHIJKLMNOPQRSTUVWXYZabcdefghijklmnopqrstuvwxyz}
\newlength{\highlightheight}
\newlength{\highlightdepth}
\newlength{\highlightmargin}
\setlength{\highlightmargin}{2pt}
\settoheight{\highlightheight}{\alphabet}
\settodepth{\highlightdepth}{\alphabet}
\addtolength{\highlightheight}{\highlightmargin}
\addtolength{\highlightdepth}{\highlightmargin}
\addtolength{\highlightheight}{\highlightdepth}
\newcommand*{\Highlight}{\rlap{\textcolor{HighlightBackground}{\rule[-\highlightdepth]{\linewidth}{\highlightheight}}}}
%%%%%%%%%%%%%%%%%%%%%%%%%%%%%%%%%%%%%%%%%%%%%%%%%%%%%%%%%%%%%
%%--------------------------------%%
\begin{withoutheadline}
\frame{
  \titlepage
}
\end{withoutheadline}

%%--------------------------------%%
\AtBeginSection[]{
\begin{frame}
  \frametitle{Outline}
  \tableofcontents[currentsection]
\end{frame}
}
\section{Background}
\subsection{My Background}
  \begin{frame}
    \frametitle{Removing assumptions in nuclear fuel cycle modeling.}
    % The through line of my research is using computational tools to remove
    % assumptions.
    % \vspace{0.6cm}
    \begin{columns}
      \column[t]{5cm}
      I am a Masters student in the Advanced Reactors and Fuel Cycles group at
      UIUC under Prof. Madicken Munk and Prof. Kathryn Huff.
      \begin{center}
              \includegraphics[height=0.2\textheight]{./images/arfc-logo}
      \end{center}

      \column[t]{5cm}
      I earned my B.S. in Engineering Physics from UIUC.
      \begin{figure}[htbp!]
        \begin{center}
          \includegraphics[height=3cm]{./images/ill_phys.png}
        \end{center}
        % \caption{A caption describing the image. \cite{lastname_firstword_1900}.}
        \label{fig:uiuc_phys}
      \end{figure}
    \end{columns}
    \begin{itemize}
      \item Making transaction models more detailed.
      \item Identifying realtime diversion or diversion paths.
      \item Making facility models more accurate.
      \item Finding advanced reactor impacts on the fuel cycle.
    \end{itemize}
  \end{frame}


\section{Nuclear Fuel Cycle}
\subsection{Fuel Cycle Overview}
  \begin{frame}
      \frametitle{Generally, fuel cycles have these steps.}
      \begin{figure}[ht!]
      \centering
      \includegraphics[width=0.75\textwidth]{images/nuclear_fuel_cycle.png}
      \caption{General nuclear fuel cycle overview \cite{penn_fc}.}
      \end{figure}
  \end{frame}

  \begin{frame}
      \frametitle{Not all fuel cycles are made equal, and we want options.}
      Concerns about economics, waste generation, proliferation risk, and sustainability motivate the need for fuel cycle options. With metrics like:
        \begin{itemize}%[<+->]
            \item natural resource utilization, % mention or-sage
            \item waste mass/volume,
            \item special material quantities,
            \item separative work units,
            \item and energy production,
        \end{itemize}
        we can begin to evaluate the tradeoffs between fuel cycle options.
  \end{frame}

  \begin{frame}
    \frametitle{We use \cyclus to model fuel cycles.}
    \vspace{20pt}
    \cyclus is an open-source agent-based fuel cycle code allowing for detailed facility and transaction modeling \cite{huff_fundamental_2016}.
    \vspace{30pt}
    \begin{figure}
        \centering
        \includegraphics[width=0.45\textwidth]{images/cyclus_logo.png}
        % mention ORION
    \end{figure}
  \end{frame}

  \begin{frame}
    \frametitle{\cyclus is being used to tackle big questions.}
    \begin{block}{Transaction Models.}
        There is active work to incorporate realistic purchasing agreements and market models into \cyclus.
    \end{block}
    \begin{block}{Nonproliferation and Safeguards.}
        CNTAUR \cite{mummah_advanced_2024} and Pyre \cite{westphal_modeling_2019} format outputs in IAEA code 10 format and model real time diversion, respectively.
    \end{block}
    \begin{block}{Facility Models.}
      OpenMCyclus \cite{openmcyclus_paper} couples \cyclus with OpenMC to model realtime depletion. From my work, we will discuss the \gls{dpr}, \gls{tod} reactor, and \gls{ever} today.
  \end{block}
    \begin{block}{Transition Scenarios.}
        We will talk about this in the context of advanced reactors.
    \end{block}
  \end{frame}

\section{Enrichment and Core Loading Versatility}
\subsection{2023 \cyclus NEUP}
\begin{frame}
  \frametitle{Illuminating Emerging Supply Chain and Waste Management
  Challenges.}
  This work is one part of a broader effort to enhance the \cyclus fuel cycle code. The three areas of work are to:
  \begin{itemize}
    \item improve modeling of supply chain dynamics,
    \item account for regional and temporal variability in material needs,
    \item and expand models appropriate for variations in reactor fueling strategies.
  \end{itemize}
\end{frame}

\begin{frame}
  \frametitle{Varying core loading improves fuel utilization.}
  Fuel cycle simulators often \cite{out_of_core} assume that:
  \begin{itemize}
    \item the utilization of each fuel assembly is the same,
    \item and the reactor power is constant over its lifetime when not refueling.
  \end{itemize}
  When we create models, these can result in separate assumptions for the reactor and fuel that are not necessarily connected.
\end{frame}

\begin{frame}
  \frametitle{EVER changes the primary fuel for a reactor.}
  \begin{figure}
    \centering
    \includegraphics[width=0.90\textwidth]{images/ever_diagram.png}
    % \caption{The idea of the Enrichment Versatile non-Equilibrium Reactor (EVER).}
  \end{figure}
\end{frame}

% \begin{frame}
%   \frametitle{single deployment}
% \end{frame}

\begin{frame}
  \frametitle{This toy scenario moves from HALEU to LEU.}
  \begin{figure}
    \centering
    \includegraphics[width=0.75\textwidth]{images/mass_fuel.png}
    \caption{The amount of fuel supplied to the reactor.}
  \end{figure}
\end{frame}

\begin{frame}
  \frametitle{The HALEU fuel is visible in the isotopics of stored fuel.}
  \begin{figure}
    \centering
    \includegraphics[width=0.75\textwidth]{images/mass_isotopes.png}
    \caption{The cumulative isotopes stored in the repository.}
  \end{figure}
\end{frame}

\subsection{Future Work}
\begin{frame}
  \frametitle{Future work on EVER.}
  \begin{itemize}
    \item Pre-generate core-averaged cross sections and update group constant data.
    \item Vary recycling technology (PUREX, Electrolysis, Pyroprocessing).
    \item Incorporate different cooling, production, and processing times according to fuel type.
    \item Introduce the ability for the user to specify the location of fuel elements in the reactor core.
  \end{itemize}
\end{frame}

\section{Dynamic Power and On-Demand Trading}

\subsection{Dynamic Power Reactor}
\begin{frame}
  \frametitle{\cycamore's capacity was unequal 62.2\% for the days.}
  \begin{figure}
    \centering
    \includegraphics[width=0.75\textwidth]{images/power_percent_clinton_fake.pdf}
    \caption{Modeling Clinton's power with the \cycamore reactor.}
  \end{figure}
\end{frame}

\begin{frame}
  \frametitle{Using NRC data, we can modify the reactor's power capacity.}
  \begin{figure}
    \centering
    \includegraphics[width=0.75\textwidth]{images/dpr_cycamore_energy.pdf}
    \caption{Comparing the DPR Clinton to the \cycamore Clinton.}
  \end{figure}
\end{frame}

\begin{frame}
  \frametitle{Max difference between Clinton and DPR is $2.22 \times 10^{-16}$.}
  \begin{figure}
    \centering
    \includegraphics[width=0.75\textwidth]{images/dpr_diff.pdf}
    \caption{Comparison of the Clinton Power Station and the Dynamic Power Reactor.}
  \end{figure}
\end{frame}


\subsection{Trading On-Demand Reactor}
\begin{frame}
  \frametitle{A \cyclus time step has 3 parts.}
  Every agent in a \cyclus simulation undergoes 5 phases:
  \begin{itemize}
    \item agents enter simulation (Building Phase)
    \item agents respond to current simulation state (Tick Phase)
    \item resource exchange execution (Exchange Phase)
    \item agents respond to current simulation state (Tock Phase)
    \item agents leave simulation (Decommissioning Phase)
  \end{itemize}
  Between the build and decommissioning phases, every agent cycles through the tick and tock phases for each universal time step.
\end{frame}

% \begin{frame}
%   \frametitle{\cyclus has a universal time step for all agents.}

% \end{frame}

\begin{frame}
  \frametitle{Using TOD resulted in a speedup of 1.038.}
  \begin{figure}
    \centering
    \includegraphics[width=0.75\textwidth]{images/time_clock_violin.pdf}
    \caption{A-B-C simulation of the \cycamore reactor and the \gls{tod} reactor.}
  \end{figure}
\end{frame}

\begin{frame}
  \frametitle{The TOD reactor has a higher utilization than \cycamore.}
  \begin{figure}
    \centering
    \includegraphics[width=0.8\textwidth]{images/ins_cyc_both.pdf}
    \caption{Number of instructions and instructions per cycle from the \gls{tod} and \cycamore reactors.}
  \end{figure}
\end{frame}

\subsection{Future Work}
\begin{frame}
  \frametitle{Future work on DPR and TOD.}
% 1. Finding other ways to reduce the number of instructions.
% 1. Investigating the exchange method, and how the complexity can be streamlined.
% 1. Applying similar logic to other standard fuel cycle archetypes.
% 1. Incorporating other variations in the fuel usage (allow reactors to model hot or cold shut down).
% 1. Generating some sort of synthetic variation data based on the traditional LWR fleet performance and extrapolating that into the future.
% 1. Attempting to create bounding cases that are an analogy to the performance of the advanced reactor designs we use in the work.
  \begin{itemize}
    \item Investigating the exchange method, and how the complexity can be streamlined.
    \item Applying similar trading on-demand and dynamic parameter logic to other standard fuel cycle archetypes where applicable.
    \item Incorporating other variations in the fuel usage (model hot or cold shut down, or off-cycle down powers).
    \item Generating synthetic power variation data based on the traditional LWR fleet performance and extrapolating that into the future.
    \item Attempting to create bounding cases that are an analogy to the performance of the advanced reactor designs we use in the work.
  \end{itemize}
\end{frame}

\section{Transition Scenarios}
  \subsection{LEU+ to HALEU}
  % \begin{frame}
  %   \frametitle{Our energy production is increasing.}
  %   \begin{figure}
  %       \centering
  %       \includegraphics[width=0.75\textwidth]{images/prim_prod_b_source.pdf}
  %       \caption{1950-2023 Primary Energy Production by source. Reproduced from \cite{mer_april_2024}.}
  %   \end{figure}
  % \end{frame}

  \begin{frame}
    \frametitle{What if we can't get HALEU to fuel these advanced reactors?}
    \vspace{-25pt}
    \begin{figure}
        \centering
        \includegraphics[width=0.96\textwidth]{images/reactor_timeline.png}
    \end{figure}
    Source: \url{inl.gov/nuclear-reactor-sustainment-and-expanded-deployment/}
    \vspace{-8pt}
    Could we use \gls{leup} while HALEU supply chains develop?
  \end{frame}

  \begin{frame}
    \frametitle{We define LEU+ as 5-10\% $^{235}$U enrichment.}
    \begin{table}[H]
        \centering
        \caption{Enrichment levels and their ranges.}
        \label{tab:enrichment_levels}
        \begin{tabular}{c c}
          \hline
          \textbf{Enrichment Level} & \textbf{Range [\%  $^{235}$U]} \\
          \hline
          Natural & $<$ 0.711 \\
          LEU & 0.711-5 \\
          LEU+ & 5-10 \\
          HALEU & 10-20 \\
          HEU & $\geq$ 20  \\
          \hline
        \end{tabular}
    \end{table}
    These are a mash-up of economic and regulatory definitions.
  \end{frame}

  \begin{frame}
    \frametitle{We use Serpent to approximate reactors with LEU+ or HALEU.}
    \begin{columns}
      \column[t]{5cm}
      \begin{figure}
        \centering
        \includegraphics[width=0.75\textwidth]{images/haleu_mmr_2blocks.inp_geom1.png}
        \caption{Top-down view of the \gls{leup} MMR core. Adapted from Bachmann \cite{bachmann_mmr_like_2023}.}
        \label{fig:td_mmr}
      \end{figure}

      \column[t]{5cm}
      \begin{figure}[htbp!]
        \begin{center}
          \includegraphics[width=0.75\textwidth]{images/htgr-mr-burn-200.inp_mesh1_bstep6.png}
        \end{center}
        \caption{Top-down view of the \gls{leup} Xe-100 core. Adapted from Richter \cite{richter_xe100_like}.}
        \label{fig:td_xe100}
      \end{figure}
    \end{columns}
  \end{frame}

  \subsection{Deployment Schemes}
  \begin{frame}
    \frametitle{We mimic real-world deployment by meeting energy demand.}
    \begin{figure}
      \centering
      \includegraphics[width=0.75\textwidth]{images/new_capacity_ng_d2.pdf}
      \caption{Historical and projected US nuclear energy if we double the capacity of nuclear energy by 2050.}
    \end{figure}
  \end{frame}

  \begin{frame}
    \frametitle{Schemes have built-in assumptions about the scenario.}
    \begin{table}[H]
      \centering
      \caption{Deployment schemes.}
      \label{tab:deployment_schemes}
      \newcommand{\ColWidth}{0.48\linewidth}
      \begin{tabular}{p{\ColWidth} p{\ColWidth}}
          \hline
          Scheme & Description \\
          \hline
          Greedy Deployment & Deploy reactors to fill demand, preferring to deploy larger capacity units first. \\
          \vspace{1.5mm}\\
          Random Deployment & Use the date and hour as seed to sample the
          reactors list randomly. \\
          \vspace{1.5mm}\\
          Initially Random, Greedy Deployment & Run the random scheme until
          a reactor bigger than the remaining capacity is proposed,
          then the greedy algorithm. \\
          \hline
      \end{tabular}
    \end{table}
  \end{frame}

  \begin{frame}
    \frametitle{Greedy reactor deployment scheme.}
    % \begin{algorithm}[H]
      \begin{algorithmic}[1]
        % \caption{Greedy Reactor Deployment Algorithm}
          \State Initialize demand
          \While{demand exists}
              \State Select the largest reactor that does not exceed demand
              \State Deploy reactors until the next reactor exceeds demand
              \State Update demand
          \EndWhile
      \end{algorithmic}
    % \end{algorithm}
  \end{frame}

  % \begin{frame}
  %   \frametitle{Random reactor deployment scheme.}
  % %   \begin{algorithm}[H]
  % %     \caption{Random Reactor Deployment Algorithm}
  %     \begin{algorithmic}[1]
  %         \State Initialize demand
  %         \While{demand exists}
  %             \State Randomly deploy a reactor that does not exceed demand
  %             \State Update demand
  %         \EndWhile
  %     \end{algorithmic}
  % %     \end{algorithm}
  % \end{frame}

  % \begin{frame}
  %   \frametitle{Random + greedy reactor deployment scheme.}
  % %     \begin{algorithm}[H]
  % %       \caption{Random + Greedy Reactor Deployment Algorithm}
  %       \begin{algorithmic}[1]
  %           \State Initialize demand
  %           \While{demand exists}
  %               \State Randomly deploy a reactor
  %               \If{demand is exceeded}
  %                   \State Remove last reactor
  %                   \If{demand still exists}
  %                       \State Select the largest reactor that does not exceed demand
  %                       \State Deploy until the next reactor exceeds demand
  %                       \State Update demand
  %                   \EndIf
  %               \EndIf
  %           \EndWhile
  %       \end{algorithmic}
  % %       \end{algorithm}
  % \end{frame}

  \begin{frame}
    \frametitle{Staggering enrichment allows the supply chain to develop.}
    \begin{figure}
        \centering
        \includegraphics[width=0.75\textwidth]{images/fresh_fuel.pdf}
    \end{figure}
  \end{frame}

  \begin{frame}
    \frametitle{The difference is on the order of hundreds of tons.}
    \begin{figure}
        \centering
        \includegraphics[width=0.75\textwidth]{images/fresh_fuel_difference.pdf}
    \end{figure}
  \end{frame}

  \begin{frame}
      \frametitle{Fuel cycle modeling is useful for enegy planning.}
      In our case, we transition from \gls{leup} to HALEU after 10 years of operation.
      \begin{itemize}
          \item For the Xe100 reactors, we need almost 315 less tons of HALEU.
          \item For the MMR reactors, we need almost 97 less tons of HALEU.
      \end{itemize}
      Next we need to characterize what the cost of this transition would be.
  \end{frame}


  % \subsection{Future Work}
  % \begin{frame}
  %     \frametitle{The metrics we used are economic precurors}
  %     \begin{itemize}
  %         \item F
  %     \end{itemize}
  % \end{frame}

  \section{Big Questions}
  \begin{frame}
    \frametitle{\textit{Revisit:} \cyclus is being used to tackle big questions.}
    \begin{block}{Making transaction models more detailed.}
        Incorporate geospatial restrictions with OR-SAGE, the cost implications of my results, and multi objective optimization with OSIER \cite{Dotson_osier}.
    \end{block}
    \begin{block}{Identifying realtime diversion or diversion paths.}
        Study tracer isotopes, such as $^{232}U$, suggested by Rhodes and Maldonado \cite{rhodes_u232}, and international collaboration on supply chain security.
    \end{block}
    \begin{block}{Making facility models more accurate.}
      Continue \gls{ever} and \gls{clover}, and improve the exchange method's efficiency. Vary the power of coupled physics reactor models.
    \end{block}
    \begin{block}{Finding advanced reactor impacts on the fuel cycle.}
      Consider how enrichment schemes, other reactor designs (including fusion), and the costs of fuel and waste management.
    \end{block}
  \end{frame}

  \begin{frame}
    \frametitle{More big questions.} % bad title
    \begin{block}{Reactor evOLutionary aLgorithm Optimizer (ROLLO)}
      Explore non-conventional geometries and fuel distributions to improve performance and safety.
    \end{block}
    % \begin{block}{ATF fuel modeling.}
    %   I have a half-baked idea of modeling triso atf fuel? Or maybe ATF fuel cycles?
    % \end{block}
    \begin{block}{UNF isotope characterization and recycling.}
      Also not really under this title, but a supply chain for them? With proposed research reactors?
    \end{block}
    \begin{block}{Nuclear grade graphite and .}
      Characterizing the supply chain, and identifying opportunities for secondary material use.
    \end{block}
  \end{frame}

  % \begin{frame}
  %   \frametitle{}
  % \end{frame}

\begin{frame}
  \frametitle{Acknowledgements}
  This research was performed, in part, using funding received from the DOE
  Office of Nuclear Energy's Nuclear Energy University Program (Project 23-29656
  DE-NE0009390) 'Illuminating Emerging Supply Chain and Waste Management
  Challenges'.
  \vspace{0.5cm}
  This research was supported in part by an appointment to the Oak Ridge
  National Laboratory Research Student Internships Program, sponsored by the U.
  S. Department of Energy and administered by the Oak Ridge Institute for
  Science and Education.
\end{frame}




%%--------------------------------%%
%%--------------------------------%%
\begin{frame}[allowframebreaks]
  \frametitle{References}
  \bibliographystyle{plain}
  {\footnotesize \bibliography{bibliography.bib} }

\end{frame}

\appendix

\begin{frame}
    \frametitle{Know how to code?}
    Consider volunteering as a TA or mentor in the Computational Resource Access NEtwork (CRANE) so we can support more students!
    \begin{figure}
        \centering
        \includegraphics[width=0.7\textwidth]{images/CRANE_logo_inverted.png}
    \end{figure}
    Go to our website: \url{https://www.cranephysics.org}
\end{frame}

%%--------------------------------%%


\end{document}
