%        File: arfc-beamer.tex
%     Created: Sun May 5 10:00 PM 2013 C
%


%\documentclass[11pt,handout]{beamer}
\documentclass[9pt]{beamer}
\usetheme[white]{Illinois}
%\title[short title]{long title}
\title[Updating Modeling Assumptions]{Updating assumptions in advanced reactor fuel cycles}
%\subtitle[short subtitle]{long subtitle}
\subtitle[Short SubTitle]{ORNL Symposium}
%\author[short name]{long name}
\author[Nathan Ryan]{Nathan Ryan\\Advanced Reactors and Fuel Cycles}
%\date[short date]{long date}
\date[03.08.2025]{March 8, 2025}
%\institution[short name]{long name}
\institute[UIUC]{University of Illinois Urbana-Champaign}

%\usepackage{bbding}
\usepackage{amsfonts}
% \usepackage{algorithm}
% \usepackage[ruled]{algorithm2e}
% \usepackage{algorithmic}
\usepackage{algpseudocode}
% \usepackage{algorithmic}
% \usepackage{array}
\usepackage{amsmath}
\usepackage{xspace}
\usepackage{graphicx}
\usepackage{subfigure}
\usepackage{booktabs} % nice rules for tables
\usepackage{microtype} % if using PDF
\usepackage{bigints}
\usepackage{caption}

\newcommand{\cycamore}{\textsc{Cycamore}\xspace}
\newcommand{\cyclus}{\textsc{Cyclus}\xspace}

\newcommand{\units}[1] {\:\text{#1}}%
\newcommand{\SN}{S$_N$}%{S$_\text{N}$}%{$S_N$}%
\DeclareMathOperator{\erf}{erf}
%I need some complimentary error funcitons...
\DeclareMathOperator{\erfc}{erfc}
%Those icons in the references are terrible looking
\setbeamertemplate{bibliography item}[text]

%%%% Acronym support

\usepackage[acronym,toc]{glossaries}
\include{acros}

\makeglossaries

%try to get rid of header on title page\dots
\makeatletter
    \newenvironment{withoutheadline}{
        \setbeamertemplate{headline}[default]
        \def\beamer@entrycode{\vspace*{-\headheight}}
    }{}
\makeatother

% \makeatother
% \setbeamertemplate{footline}
% {
%   \leavevmode%
%   \hbox{%
%     \rightline{\insertframenumber{} / \inserttotalframenumber\hspace*{1ex}}
%   }%
%   \vskip0pt%
% }
% \makeatletter
\setbeamertemplate{caption}{\raggedright\insertcaption\par}
\setbeamertemplate{page number in head/foot}[appendixframenumber]

\begin{document}
%%%%%%%%%%%%%%%%%%%%%%%%%%%%%%%%%%%%%%%%%%%%%%%%%%%%%%%%%%%%%
%% From uw-beamer Here's a handy bit of code to place at
%% the beginning of your presentation (after \begin{document}):
\newcommand*{\alphabet}{ABCDEFGHIJKLMNOPQRSTUVWXYZabcdefghijklmnopqrstuvwxyz}
\newlength{\highlightheight}
\newlength{\highlightdepth}
\newlength{\highlightmargin}
\setlength{\highlightmargin}{2pt}
\settoheight{\highlightheight}{\alphabet}
\settodepth{\highlightdepth}{\alphabet}
\addtolength{\highlightheight}{\highlightmargin}
\addtolength{\highlightdepth}{\highlightmargin}
\addtolength{\highlightheight}{\highlightdepth}
\newcommand*{\Highlight}{\rlap{\textcolor{HighlightBackground}{\rule[-\highlightdepth]{\linewidth}{\highlightheight}}}}
%%%%%%%%%%%%%%%%%%%%%%%%%%%%%%%%%%%%%%%%%%%%%%%%%%%%%%%%%%%%%
%%--------------------------------%%
\begin{withoutheadline}
\frame{
  \titlepage
}
\end{withoutheadline}

%%--------------------------------%%
\AtBeginSection[]{
\begin{frame}
  \frametitle{Outline}
  \tableofcontents[currentsection]
\end{frame}
}

\section{My Background}
  \begin{frame}
    \frametitle{Removing assumptions in nuclear fuel cycle modeling}
    % The through line of my research is using computational tools to remove
    % assumptions.
    % \vspace{0.6cm}
    \begin{columns}
      \column[t]{5cm}
      I am a Masters student in the Advanced Reactors and Fuel Cycles group at
      UIUC under Professors Madicken Munk and Kathryn Huff.
      \begin{center}
              \includegraphics[height=0.2\textheight]{./images/arfc-logo}
      \end{center}

      \column[t]{5cm}
      I earned my B.S. in Engineering Physics from UIUC.
      \begin{figure}[htbp!]
        \begin{center}
          \includegraphics[height=3cm]{./images/ill_phys.png}
        \end{center}
        % \caption{A caption describing the image. \cite{lastname_firstword_1900}.}
        \label{fig:uiuc_phys}
      \end{figure}
    \end{columns}
  \end{frame}


\section{Nuclear Fuel Cycle}
\subsection{Fuel Cycle Overview}
  \begin{frame}
      \frametitle{Generally, fuel cycles have these steps.}
      \begin{figure}[ht!]
      \centering
      \includegraphics[width=0.75\textwidth]{images/nuclear_fuel_cycle.png}
      \caption{Source: Penn State Radiation Science and Engineering Center (public domain)$^{*}$}
      \end{figure}
  \end{frame}

  \begin{frame}
      \frametitle{Not all fuel cycles are made equal, and we want options.}
      Concerns about economics, waste generation, proliferation risk, and sustainability motivate the need for fuel cycle options. With metrics like:
        \begin{itemize}%[<+->]
            \item natural resource utilization, % mention or-sage
            \item waste mass/volume,
            \item special material quantities,
            \item separative work units,
            \item and energy production,
        \end{itemize}
        we can begin to evaluate the tradeoffs between fuel cycle options.
  \end{frame}

\subsection{Fuel Cycle Modeling}
  \begin{frame}
    \frametitle{We use \cyclus to model fuel cycles.}
    \vspace{20pt}
    \cyclus is an open-source agent-based fuel cycle code allowing for detailed facility and transaction modeling \cite{huff_fundamental_2016}.
    \vspace{20pt}
    \begin{figure}
        \centering
        \includegraphics[width=0.45\textwidth]{images/cyclus_logo.png}
        % mention ORION
    \end{figure}

    \vspace{37pt}
    Source: \url{https://github.com/cyclus/cyclus.github.com/blob/source/source/logos/logo2_transp.png}
  \end{frame}

  \begin{frame}
    \frametitle{\cyclus is being used to tackle big questions.}
    \begin{block}{Making transaction models more detailed.}
        There is active work to incorporate realistic purchasing agreements and market models into \cyclus.
    \end{block}
    \begin{block}{Identifying realtime diversion or diversion paths.}
        CNTAUR \cite{mummah_advanced_2024} and Pyre \cite{westphal_modeling_2019} format outputs in IAEA code 10 format and model real time diversion, respectively.
    \end{block}
    \begin{block}{Making facility models more accurate.}
      OpenMCyclus \cite{openmcyclus_paper} couples \cyclus with OpenMC to model realtime depletion. From my work, we will discuss the \gls{dpr}, \gls{tod} reactor, \gls{ever}, and \gls{clover} today.
  \end{block}
    \begin{block}{Finding advanced reactor impacts on the fuel cycle.}
        We will talk about this in the context of transition scenarios.
    \end{block}
  \end{frame}

\section{Enrichment and Core Loading Versatility}
\begin{frame}
  \frametitle{NEUP goals.}
  This work is one part of a broader effort to enhance the \cyclus fuel cycle code. The three areas of work are to:
  \begin{itemize}
    \item improve modeling of supply chain dynamics,
    \item account for regional and temporal variability in material needs,
    \item and expand models appropriate for variations in reactor fueling strategies.
  \end{itemize}
\end{frame}

\begin{frame}
  \frametitle{Varying core loading and burnup improves accuracy.}
  Fuel cycle simulators often \cite{out_of_core} assume that:
  \begin{itemize}
    \item the burnup of each fuel assembly is the same,
    \item and each fuel element is exposed to the same spectrum on average over its lifetime.
  \end{itemize}
  These are two sides of the same coin, but, when we create models, can result in separate assumptions for the reactor and fuel that are not necessarily connected.
\end{frame}

\begin{frame}
  \frametitle{EVER changes the primary fuel for a reactor.}
  \begin{figure}
    \centering
    \includegraphics[width=0.90\textwidth]{images/ever_diagram.png}
    % \caption{The idea of the Enrichment Versatile non-Equilibrium Reactor (EVER).}
  \end{figure}
\end{frame}

% \begin{frame}
%   \frametitle{single deployment}
% \end{frame}

\begin{frame}
  \frametitle{This toy scenario moves from HALEU to LEU.}
  \begin{figure}
    \centering
    \includegraphics[width=0.75\textwidth]{images/mass_fuel.png}
    \caption{The amount of fuel supplied to the reactor.}
  \end{figure}
\end{frame}

\begin{frame}
  \frametitle{The HALEU fuel is visible in the isotopics of stored fuel.}
  \begin{figure}
    \centering
    \includegraphics[width=0.75\textwidth]{images/mass_isotopes.png}
    \caption{The cumulative isotopes stored in the repository.}
  \end{figure}
\end{frame}

\begin{frame}
  \frametitle{Future work on EVER.}
  \begin{itemize}
    \item Pre-generate core-averaged cross sections and update group constant data.
    \item Vary recycling technology (PUREX, Electrolysis, Pyroprocessing).
    \item Incorporate different cooling, production, and processing times according to fuel type.
    \item Introduce the ability for the user to specify the location of fuel elements in the reactor core.
  \end{itemize}
\end{frame}

\section{Dynamic Power and On-Demand Trading}

\subsection{Dynamic Power Reactor}
\begin{frame}
  \frametitle{\cycamore's power was unequal 62.2\% of the 1460-day simulation.}
  \begin{figure}
    \centering
    \includegraphics[width=0.75\textwidth]{images/power_percent_clinton_fake.pdf}
    \caption{Modeling Clinton's power with the \cycamore reactor.}
  \end{figure}
\end{frame}

\begin{frame}
  \frametitle{Using NRC data, we can modify the reactor's power capacity.}
  \begin{figure}
    \centering
    \includegraphics[width=0.75\textwidth]{images/dpr_cycamore_energy.pdf}
    \caption{Comparing the DPR Clinton to the \cycamore Clinton.}
  \end{figure}
\end{frame}

\begin{frame}
  \frametitle{Max difference between Clinton and DPR is $2.22 \times 10^{-16}$.}
  \begin{figure}
    \centering
    \includegraphics[width=0.75\textwidth]{images/dpr_diff.pdf}
    \caption{Comparison of the Clinton Power Station and the Dynamic Power Reactor.}
  \end{figure}
\end{frame}


\subsection{Trading On-Demand Reactor}
\begin{frame}
  \frametitle{intro tod}
\end{frame}

\begin{frame}
  \frametitle{t}
\end{frame}

\subsection{Future Work}
\begin{frame}
  \frametitle{Future work on DPR and TOD.}
\end{frame}

\section{Transition Scenarios}
  \subsection{Deployment Schemes}
  \begin{frame}
    \frametitle{We mimic real-world deployment by meeting energy demand.}
    \begin{figure}
      \centering
      \includegraphics[width=0.75\textwidth]{images/new_capacity_ng_d2.pdf}
      \caption{Historical and projected US nuclear energy if we double the capacity of nuclear energy by 2050.}
    \end{figure}
  \end{frame}

  \begin{frame}
    \frametitle{Greedy reactor deployment scheme.}
    % \begin{algorithm}[H]
      \begin{algorithmic}[1]
        % \caption{Greedy Reactor Deployment Algorithm}
          \State Initialize demand
          \While{demand exists}
              \State Select the largest reactor that does not exceed demand
              \State Deploy reactors until the next reactor exceeds demand
              \State Update demand
          \EndWhile
      \end{algorithmic}
    % \end{algorithm}
  \end{frame}

  \begin{frame}
    \frametitle{Random reactor deployment scheme.}
  %   \begin{algorithm}[H]
  %     \caption{Random Reactor Deployment Algorithm}
      \begin{algorithmic}[1]
          \State Initialize demand
          \While{demand exists}
              \State Randomly deploy a reactor that does not exceed demand
              \State Update demand
          \EndWhile
      \end{algorithmic}
  %     \end{algorithm}
  \end{frame}

  \begin{frame}
    \frametitle{Random + greedy reactor deployment scheme.}
  %     \begin{algorithm}[H]
  %       \caption{Random + Greedy Reactor Deployment Algorithm}
        \begin{algorithmic}[1]
            \State Initialize demand
            \While{demand exists}
                \State Randomly deploy a reactor
                \If{demand is exceeded}
                    \State Remove last reactor
                    \If{demand still exists}
                        \State Select the largest reactor that does not exceed demand
                        \State Deploy until the next reactor exceeds demand
                        \State Update demand
                    \EndIf
                \EndIf
            \EndWhile
        \end{algorithmic}
  %       \end{algorithm}
  \end{frame}

  \subsection{LEU+ to HALEU}
  \begin{frame}
    \frametitle{What if we can't get HALEU to fuel these advanced reactors?}
    \vspace{-25pt}
    \begin{figure}
        \centering
        \includegraphics[width=0.98\textwidth]{images/reactor_timeline.png}
        \caption{Source: \url{inl.gov/nuclear-reactor-sustainment-and-expanded-deployment/}}
    \end{figure}
    \vspace{-8pt}
    Could we use LEU+ in the meantime?
  \end{frame}

  \begin{frame}
    \frametitle{We define LEU+ as 5-10\% $^{235}$U enrichment.}
    \begin{table}[H]
        \centering
        \caption{Enrichment levels and their ranges.}
        \label{tab:enrichment_levels}
        \begin{tabular}{c c}
           \hline
           \textbf{Enrichment Level} & \textbf{Range [\%  $^{235}$U]} \\
           \hline
           Natural & $<$ 0.711 \\
           LEU & 0.711-5 \\
           LEU+ & 5-10 \\
           HALEU & 10-20 \\
           HEU & $\geq$ 20  \\
           \hline
        \end{tabular}
     \end{table}
     These are a mash-up of economic and regulatory definitions.
  \end{frame}

  \begin{frame}
    \frametitle{Our energy production is increasing.}
    \begin{figure}
        \centering
        \includegraphics[width=0.75\textwidth]{images/prim_prod_b_source.pdf}
        \caption{1950-2023 Primary Energy Production by source \cite{mer_april_2024}.}
    \end{figure}
  \end{frame}

  \begin{frame}
    \frametitle{Staggering enrichment allows the supply chain to develop.}
    \begin{figure}
        \centering
        \includegraphics[width=0.75\textwidth]{images/fresh_fuel.pdf}
    \end{figure}
  \end{frame}

  \begin{frame}
    \frametitle{The difference is on the order of hundreds of tons.}
    \begin{figure}
        \centering
        \includegraphics[width=0.75\textwidth]{images/fresh_fuel_difference.pdf}
    \end{figure}
  \end{frame}


  \subsection{Conclusion}
  \begin{frame}
      \frametitle{Fuel cycle modeling is useful for enegy planning and safeguards.}
      We have covered a tiny fraction of what fuel cycle modeling can do, but there is so much more to do. In our simple case, we transition from LEU+ to HALEU after 10 years of operation.
      \begin{itemize}
          \item For the Xe100 reactors, we need almost 315 less tons of HALEU.
          \item For the MMR reactors, we need almost 97 less tons of HALEU.
      \end{itemize}
      Next we need to characterize what the cost of this transition would be.
  \end{frame}


  \subsection{Future Work}
  \begin{frame}
      \frametitle{t}
      \begin{itemize}
          \item F
      \end{itemize}
  \end{frame}

  \section{Big Questions}
  \begin{frame}
    \frametitle{\textit{Revisit:} \cyclus is being used to tackle big questions.}
    \begin{block}{Making transaction models more detailed.}
        Incorporate geospatial restrictions with OR-SAGE, and optimization across multiple objectives with OSIER \cite{Dotson_osier}.
    \end{block}
    \begin{block}{Identifying realtime diversion or diversion paths.}
        Explore the accumulation and detection of tracer isotopes such as $^{232}U$, as suggested by Rhodes and Maldonado \cite{rhodes_u232}. Additionally, expand the coupled physics reactor models to include variation in power.
    \end{block}
    \begin{block}{Making facility models more accurate.}
      Continue to develop \gls{ever}, and \gls{clover}, and explore how we can improve the computational efficiency of the exchange method in \cyclus
    \end{block}
    \begin{block}{Finding advanced reactor impacts on the fuel cycle.}
      Consider how enrichment schemes, other reactor designs, and international collaboration affect costs of fuel and waste management.
    \end{block}
  \end{frame}

  % \begin{frame}
  %   \frametitle{}
  % \end{frame}

\begin{frame}
  \frametitle{Acknowledgements}
  This research was performed, in part, using funding received from the DOE
  Office of Nuclear Energy's Nuclear Energy University Program (Project 23-29656
  DE-NE0009390) 'Illuminating Emerging Supply Chain and Waste Management
  Challenges'.
  \vspace{0.5cm}
  This research was supported in part by an appointment to the Oak Ridge
  National Laboratory Research Student Internships Program, sponsored by the U.
  S. Department of Energy and administered by the Oak Ridge Institute for
  Science and Education.
\end{frame}




%%--------------------------------%%
%%--------------------------------%%
\begin{frame}[allowframebreaks]
  \frametitle{References}
  \bibliographystyle{plain}
  {\footnotesize \bibliography{bibliography.bib} }

\end{frame}

\appendix

\begin{frame}
    \frametitle{Know how to code?}
    Consider volunteering as a TA or mentor in the Computational Resource Access NEtwork (CRANE) so we can support more students!
    \begin{figure}
        \centering
        \includegraphics[width=0.7\textwidth]{images/CRANE_logo_inverted.png}
    \end{figure}
    Go to our website: \url{https://www.cranephysics.org}
\end{frame}

%%--------------------------------%%


\end{document}
